%\enablemode[draft]
\enablemode[attributions]
\enablemode[indices]
%\enablemode[preprint]

%\showstruts

%
% {{{1 basic setup
%
\enableregime[utf]
\mainlanguage[pl]
\setupexternalfigures[directory={ilustr}]
\tolerance=2000

\enabletrackers[fonts.missing]

\setupinteraction[
    title={Śpiewnik kolęd},
%   subtitle={},
    author={Stanisław Porczyk},
    state=start,
    color=black,
    contrastcolor=black,
]

\setupexport[
    firstpage={ilustr/narodzenie/166 - The Nativity of Our Lord.gif},
]

\setupbackend[
%   export=spiewnik.xml,
%   xhtml=spiewnik.xhtml,
%   css=spiewnik.css,
    format={pdf/a-1a:2005},
    profile={default_cmyk.icc,default_rgb.icc,default_gray.icc},
    intent={ISO coated v2 300\letterpercent\space (ECI)},
]

\setupstructure[state=start,method=auto]

%\setupinteractionscreen[option=bookmark]
\placebookmarks[booktitle,title,section,subject][force=yes]

%
% layout, dimensions {{{1
%
\doifmodeelse{preprint}{
\setuppapersize[A5]
}{
\setuppapersize[A5][A4,landscape]
\setuparranging[2UP]
}

\setuplayout[location={middle,middle}]
% 148 x 210 mm
\setuplayout[
    % w pionie
    topspace=9mm,
    height=193mm, % nie ruszać
    header=0mm,
    headerdistance=0mm,
    footerdistance=2.5mm,
    footer=5mm,
    top=5mm,
    %
    width=110mm, % nie ruszać
    backspace=18mm,
    margin=7mm,
    margindistance=3mm,
    edgedistance=1mm,
    ]
%\showframe


\setupitemgroup[itemize][inbetween=,before=,after=]
\setupitemgroup[itemize][1][n][style=em,width=8mm]
%\setupitemgroup[itemize][2][]
%\setupitemgroup[itemize][3][n]

\setupnarrower[left=8mm,right=0mm]
\setupwhitespace[none]
\definelines[stanza][
    option=packed,
    before=,
    after={\strut\vskip 3mm},
    inbetween=,
    align=normal]
\definelines[refrain][
    option=packed,
    before={\unskip\startnarrower},
    after={\stopnarrower\vskip 3mm},
    inbetween=,
    align=normal]

%
% {{{1 fonts, styles
%

%\usemodule[simplefonts]

\definefontfeature[default][default][protrusion=quality,expansion=quality,script=latn]

\starttypescriptcollection[linuxlibertine]

    \starttypescript [serif] [linuxlibertine]
        \definefontsynonym [Libertine-Regular]    [file:LinLibertine_R.otf]
        \definefontsynonym [Libertine-Italic]     [file:LinLibertine_RI.otf]
        \definefontsynonym [Libertine-Bold]       [file:LinLibertine_RB.otf]
        \definefontsynonym [Libertine-BoldItalic] [file:LinLibertine_RBI.otf]
    \stoptypescript

    \starttypescript [serif] [linuxlibertine] [name]
        \setups[font:fallback:serif]
        \definefontsynonym [Serif]           [Libertine-Regular]    [features=default]
        \definefontsynonym [SerifItalic]     [Libertine-Italic]     [features=default]
        \definefontsynonym [SerifBold]       [Libertine-Bold]       [features=default]
        \definefontsynonym [SerifBoldItalic] [Libertine-BoldItalic] [features=default]
        \definefontsynonym [SerifCaps]       [Libertine-Regular]    [features=smallcaps]
    \stoptypescript

    \starttypescript [sans] [biolinum]
        \setups[font:fallback:sans]
        \definefontsynonym [Biolinum-Regular]    [file:LinBiolinum_R.otf]
        \definefontsynonym [Biolinum-Bold]       [file:LinBiolinum_RB.otf]
        \definefontsynonym [Biolinum-Italic]     [file:LinBiolinum_RI.otf]
%       \definefontsynonym [Biolinum-Slanted]    [file:LinBiolinum_R.otf]
        \definefontsynonym [Biolinum-BoldItalic] [file:LinBiolinum_RBI.otf]
    \stoptypescript

    \starttypescript [sans] [biolinum] [name]
        \setups[font:fallback:sans]
        \definefontsynonym [Sans]           [Biolinum-Regular]    [features=default]
        \definefontsynonym [SansBold]       [Biolinum-Bold]       [features=default]
        \definefontsynonym [SansItalic]     [Biolinum-Italic]     [features=default]
%       \definefontsynonym [SansSlanted]    [Biolinum-Slanted]    [features=default]
        \definefontsynonym [SansBoldItalic] [Biolinum-BoldItalic] [features=default]
        \definefontsynonym [SansCaps]       [Biolinum-Regular]    [features=smallcaps]
    \stoptypescript

    \starttypescript [linuxlibertine]
        \definetypeface [linuxlibertine] [rm] [serif] [linuxlibertine] [default]
        \definetypeface [linuxlibertine] [ss] [sans]  [biolinum]  [default]
        \definetypeface [linuxlibertine] [tt] [mono]  [default]   [default]
        %definetypeface [libertine] [mm] [math]  [times]     [default]
        \quittypescriptscanning
    \stoptypescript

\stoptypescriptcollection


\usetypescript[linuxlibertine]
\setupbodyfont[linuxlibertine]
%\setupbodyfont[palatino]

\setupalign[hz,hanging]

%\definebodyfontenvironment[default][xx=8pt,x=10pt,text=11pt,a=12pt,b=14pt]



%\setupfootertexts[\setups{footer}]
%\setupheadertexts[\setups{headero}][][\setups{headere}][]
%\setupfootertexts[\setups{footer}]
\setupfootertexts[\setups{footero}][][\setups{footere}][]
\setuppagenumbering[location=,alternative=doublesided]

\startsetups[footer]
\hfill---~\pagenumber~---\hfill
\stopsetups

\startsetups[footero]
\hfill\bf\pagenumber
\stopsetups

\startsetups[footere]
\bf\pagenumber\hfill
\stopsetups

\startsetups[headero]
\underbar{{\tfx Śpiewnik kolęd, 2015\doifmode{draft}{\hfill\color[red]{PROJEKT}}\hfill}\bf\pagenumber}
\stopsetups

\startsetups[headere]
\underbar{{\bf\pagenumber}\tfx\doifmode{draft}{\hfill\color[red]{PROJEKT}}\hfill Śpiewnik kolęd, 2015}
\stopsetups


\setupdelimitedtext[quotation:1][left=«,right=»]
\setupdelimitedtext[quote:1][left=„,right=”]

\startsetups[table:attributions]
% attributions
\setupTABLE[each][each][style={\tfx\em},frame=off]
\setupTABLE[r][each][height=4.5mm]
%\setupTABLE[c][1][width=15mm,align={flushright,lohi},style={\tfx\em\bf}]
\setupTABLE[c][1][align={flushright,lohi},style={\tfx\ss\bf}]
\setupTABLE[c][2][width=broad]
\stopsetups

\startsetups[table:dedication]
\setupTABLE[frame=off]
\setupTABLE[c][1][width=40mm]
\setupTABLE[c][2][style={\tfa\em},align={flushleft,lohi}]
\stopsetups


\useexternalfigure[fillersmall]%
    [śnieg/Ice_cristal_-_heraldic_figure]%
    [align=middle,width=15mm,]

\def\fillersmall{%
\vskip 0pt plus 8fill%
\midaligned{\externalfigure[fillersmall]}%
\vskip 0pt plus 5fill%
}

%
%
% titles, indices {{{1
%

\definestructureresetset[default][1,1,0][1]
\setuphead[sectionresetset=default]

% kolęda
\setuphead[section][
    style={\ss\tfa\bf},
%   numberstyle={\ss\tfb\bf},
    alternative=inmargin,
%   before={\testpage[7]\blank[3mm]},
    after={\blank[3mm]},
    page=yes,
    sectionsegments=section,
]

% spis treści
\setuphead[subject][ % spis treści etc.
    page=no,
    before=,
    after=,
    style={\ss\tfa\bf},
    align=middle,
    incrementnumber=no,
    number=no,
]

% strona tytułowa (dodatku również)
\def\titlecmd#1{\switchtobodyfont[palatino]\tfd\bf #1}
\setuphead[title][
    before={\blank[60mm]},
    after={\blank[20mm]},
    textcommand=\titlecmd,
    alternative=middle,
    incrementnumber=yes,
    number=no,
    page=no,
]

\definehead[booktitle][title]

\definecombinedlist[content][title,section]
%\definecombinedlist[content][subject,section]
\setupcombinedlist[content][criterium=all,alternative=c]

\setuplist[interaction=all,color=black,distance=-5mm]
%\setuplist[title][headnumber=no]

\defineprocessor[sc][style=sc]

\defineregister[author]
\setupregister[author][
    n=1,
    balance=no,indicator=no]
\defineregister[composer]
\setupregister[composer][
    n=1,
    balance=no,indicator=no,
%   pagenumber=no,number=yes
    ]

%}}}

\starttext

% Śpiewnik kolęd {{{1
\setupheader[state=high]
\setupfooter[state=high]
\strut
\startbooktitle[title={Śpiewnik kolęd}]
%\placefigure[none]{}{\externalfigure[znak_roku_2014][width=0.6\textwidth]}
\vfill

\startlines[align=middle]
\doifmode{draft}{\color[red]{PROJEKT --- NIE DO DRUKU}}

{\sc Boże Narodzenie 2015}
\stoplines

\page[yes]
\setupheader[state=high]
\setupfooter[state=high]

\strut
\vfill
{\tfx\em oprac.~rodzina Porczyków}

\page[yes]
\setupheader[state=high]
\setupfooter[state=high]

\startalignment[middle]

\strut\vskip 0pt plus 3fill

\start
\setups[table:dedication]
\bTABLE
\bTR\bTD\eTD\bTD Dla naszych Dziadków,\eTD\eTD
\bTR\bTD\eTD\bTD Krystyny~i~Edwarda~Józefowiczów\eTD\eTR
\eTABLE
\stop

\vskip 0pt plus 1fill
\strut

\stopalignment

\setups[table:attributions]

\page[empty]

\subject{Spis treści}
\placecontent

\fillersmall

\page[yes,odd]

\startsection[title={Ach, ubogi żłobie}] %{{{1
% http://bibliotekapiosenki.pl/Ach_ubogi_zlobie
%\author{Mioduszewski, ks. Michał, CM}
\author{Siedlecki, ks. Jan, CM}
\composer{Studziński, Piotr}

\startstanza
Ach ubogi żłobie, cóż ja widzę w~Tobie?
Droższy widok, niż ma niebo, w~maleńkiej Osobie.\stopstanza

\startstanza
Zbawicielu drogi, takżeś to ubogi;
opuściłeś śliczne niebo, obrałeś barłogi.\stopstanza

\startstanza
Czyżeś nie mógł sobie w~największej ozdobie
obrać pałacu drogiego, nie w~tym leżeć żłobie?\stopstanza

\startstanza
Gdy na świat przybywasz, grzechy z~niego zmywasz,
a~na zmycie tej sprośności, gorzkie łzy wylewasz.\stopstanza

\startstanza
Na twarz upadamy, czołem uderzamy,
witając Cię w~tej stajence między bydlętami.\stopstanza

\vfill
\bTABLE
    \bTR\bTD t.:\eTD\bTD XVIII~w.\eTD\eTR
    \bTR\bTD m.:\eTD\bTD Piotr Studziński (XIX~w.)\eTD\eTR
    \bTR\bTD\eTD\bTD w~\quote{Śpiewniczku} ks.~Jana Siedleckiego CM (1878~r.)\eTD\eTR
\eTABLE


\startsection[title={Anioł pasterzom mówił}] %{{{1
% http://bibliotekapiosenki.pl/Aniol_pasterzom_mowil
\author{Kancjonał kórnicki}

\startstanza
Anioł pasterzom mówił:
Chrystus się nam narodził
w~Betlejem, nie bardzo podłym mieście,
narodził się w~ubóstwie
Pan wszego stworzenia.\stopstanza

\startstanza
Chcąc się dowiedzieć tego,
poselstwa wesołego
bieżeli do Betlejem skwapliwie,
znaleźli Dziecię w~żłobie,
Maryję z~Józefem.\stopstanza

\startstanza
Taki Pan chwały wielkiej
uniżył się Wysoki,
pałacu kosztownego żadnego
nie miał zbudowanego
Pan wszego stworzenia.\stopstanza

\startstanza
O~dziwne narodzenie,
nigdy nie wysłowione!
Poczęła Panna Syna w~czystości,
porodziła w~całości
panieństwa swojego.\stopstanza

\startstanza
Już się ono spełniło,
co pod figurą było:
Arona różdżka ona zielona
stała się nam kwitnąca
i~owoc rodząca.\stopstanza

\page[yes]

\startstanza
Słuchajcież Boga Ojca,
jako wam Go zaleca:
Ten ci jest Syn najmilszy, jedyny,
w~raju wam obiecany,
Tego wy słuchajcie.\stopstanza

\startstanza
Bogu bądź cześć i~chwała,
która by nie ustała,
jak  Ojcu, tak i~Jego Synowi,
i~Świętemu Duchowi,
w~Trójcy jedynemu.\stopstanza

\fillersmall
\bTABLE
    \bTR\bTD t.:\eTD\bTD XVI~w., \quote{Kancjonał kórnicki} (1551-1555~r.)\eTD\eTR
    \bTR\bTD m.:\eTD\bTD XVII~w.\eTD\eTR
\eTABLE


\startsection[title={A~cóż z~tą Dzieciną}] %{{{1
% http://bibliotekapiosenki.pl/A_coz_z_ta_Dziecina
\author{franciszkanki}
\author{Mioduszewski, ks. Michał, CM+„Pastorałki i kolędy”}

\startstanza
A~cóż z~tą Dzieciną będziem czynili,
pastuszkowie mili, że się nam kwili?
Zaśpiewajmy Mu wesoło
i~obróćmy się z~Nim w~koło,
hoc hoc hoc hoc hoc.\stopstanza

\startstanza
Podobno Dzieciątko, że głodne, płacze,
dlatego tak z~nami nierado skacze,
więc ja Mu dam kukiełeczkę
i~masełka osełeczkę,
pa pa pa pa pa.\stopstanza

\startstanza
Albo pacholęciu w~dudki zagrajmy
i~na piszczałeczkach rozweselajmy:
li li li li, moje dudki,
skacz, robaczku mój malutki,
li li li li li.\stopstanza

\startstanza
Jużci nie chce płakać Dziecina dłużej,
ale ukojone oczęta mruży.
Więc Go włóżmy w~kolebeczkę,
zaśpiewajmy Mu piosneczkę,
lu lu lu lu lu.\stopstanza

\vfill
\bTABLE
    \bTR\bTD t.:\eTD\bTD w~kancjonałach franciszkańskich (XVIII~w.)\eTD\eTR
    \bTR\bTD m.:\eTD\bTD \quote{Pastorałki i~kolędy}, ks. Michał Mioduszewski CM (1843~r.)\eTD\eTR
\eTABLE


\startsection[title={A~wczora z~wieczora\\(16.~symfonia anielska)}, %{{{1
    bookmark={A wczora z wieczora}]
% http://bibliotekapiosenki.pl/A_wczora_z_wieczora
% 16. symfonia anielska
\author{Żabczyc, Jan}
\author{Mioduszewski, ks. Michał, CM+„Pastorałki i kolędy”}

\startstanza
A~wczora z~wieczora,
z~niebieskiego dwora
przyszła nam nowina;
Panna rodzi Syna.\stopstanza

\startstanza
Boga prawdziwego,
nieogarnionego,
za wyrokiem Boskim,
w~Betlejem żydowskim.\stopstanza

\startstanza
Pastuszkowie mali,
w~polu w~ten czas spali,
gdy Anioł w~pół nocy
światłość z~nieba toczy.\stopstanza

\startstanza
Chwałę oznajmując,
szopę pokazując,
gdzie Panna z~Dzieciątkiem,
z~wołem i~oślątkiem.\stopstanza

\startstanza
I~Józefem starym,
nad Jezusem małym,
chwalą Boga swego
dziś narodzonego.\stopstanza

\startstanza
Witaj Królu nowy,
Synu Dawidowy,
Ty nas masz wybawić
i~w~niebie postawić.\stopstanza

\vfill
\bTABLE
    \bTR\bTD t.:\eTD\bTD \quote{Symfonie anielskie}, Jan Żabczyc (1630~r.)\eTD\eTR
    \bTR\bTD m.:\eTD\bTD \quote{Pastorałki i~kolędy}, ks. Michał Mioduszewski CM (1843 r.)\eTD\eTR
\eTABLE


\startsection[title={Bóg się rodzi}] %{{{1
% http://bibliotekapiosenki.pl/Bog_sie_rodzi_%28koleda%29
\author{Karpiński, Franciszek}
\author{Mioduszewski, ks. Michał, CM+„Śpiewnik kościelny”}

\startstanza
Bóg się rodzi, moc truchleje,
Pan niebiosów obnażony;
ogień krzepnie, blask ciemnieje,
ma granice Nieskończony.
Wzgardzony, okryty chwałą,
śmiertelny Król nad wiekami;\stopstanza
\startrefrain
    a~Słowo Ciałem się stało
    i~mieszkało między nami.\stoprefrain

\startstanza
Cóż masz niebo nad ziemiany?
Bóg porzucił szczęście swoje,
wszedł między lud ukochany,
dzieląc z~nim trudy i~znoje.
Niemało cierpiał, niemało,
żeśmy byli winni sami,\stopstanza
\startrefrain
    a~Słowo Ciałem się stało
    i~mieszkało między nami.\stoprefrain

\startstanza
W~nędznej szopie urodzony,
żłób mu za kolebkę dano.
Cóż jest, czym był otoczony?
Bydło, pasterze i~siano.
Ubodzy, was to spotkało
witać Go przed bogaczami,\stopstanza
\startrefrain
    a~Słowo Ciałem się stało
    i~mieszkało między nami.\stoprefrain

\page[yes]

\startstanza
Potem i~króle widziani
cisną się między prostotą,
niosąc dary Panu w~dani:
mirrę, kadzidło i~złoto.
Bóstwo to razem zmieszało
z~wieśniaczymi ofiarami,\stopstanza
\startrefrain
    a~Słowo Ciałem się stało
    i~mieszkało między nami.\stoprefrain

\startstanza
Podnieś rękę, Boże Dziecię,
błogosław ojczyznę miłą,
w~dobrych radach, w~dobrym bycie
wspieraj jej siłę swą siłą,
dom nasz i~majętność całą,
i~wszystkie wioski z~miastami.\stopstanza
\startrefrain
    a~Słowo Ciałem się stało
    i~mieszkało między nami.\stoprefrain

\vfill
\bTABLE
    \bTR\bTD t.:\eTD\bTD Franciszek~Karpiński (ok.~1790~r.)\eTD\eTR
    \bTR\bTD m.:\eTD\bTD \quote{Śpiewnik kościelny}, ks. Michał Mioduszewski CM (1838~r.)\eTD\eTR
\eTABLE


\startsection[title={Bracia, patrzcie jeno}] %{{{1
% http://bibliotekapiosenki.pl/Bracia_patrzcie_jeno
\author{Klonowski, Teofil}

\startstanza
Bracia patrzcie jeno, jak niebo goreje!
Znać, że coś dziwnego w~Betlejem się dzieje.
Rzućmy budy, warty, stada,
niechaj nimi Pan Bóg włada,
a~my do Betlejem, do Betlejem.\stopstanza

\startstanza
Patrzcie, jak tam gwiazda światłem swoim miga!
Pewnie do uczczenia Pana swego ściga.
Krokiem śmiałym i~wesołym
śpieszmy i~uderzmy czołem;
przed Panem w~Betlejem, w~Betlejem.\stopstanza

\startstanza
Wszakże powiedziałem, że cuda ujrzymy:
Dziecię, Boga świata, w~żłobie zobaczymy.
Patrzcie, jak biednie okryte
w~żłobie Panię znakomite,
w~szopie przy Betlejem, przy Betlejem.\stopstanza

\vfill
\bTABLE
    \bTR\bTD t. i~m.:\eTD\bTD \quote{Zbiór pieśni z~melodyjami}, Teofil Klonowski (1858~r.)\eTD\eTR
\eTABLE


\startsection[title={Cicha noc}] %{{{1
% http://bibliotekapiosenki.pl/Cicha_noc_%28koleda%29
\author{Mohr, ks. Joseph}
\author{Maszyński, Piotr}
\composer{Gruber, Franz Xaver}

\startstanza
Cicha noc, święta noc,
pokój niesie ludziom wszem,
a~u~żłobka Matka Święta
czuwa sama uśmiechnięta,
nad Dzieciątka snem.\stopstanza

\startstanza
Cicha noc, święta noc,
pastuszkowie od swych trzód
biegną wielce zadziwieni,
za anielskim głosem pieni,
gdzie się spełnił cud.\stopstanza

\startstanza
Cicha noc, święta noc,
narodzony Boży Syn,
Pan wielkiego majestatu,
niesie dziś całemu światu
odkupienie win.\stopstanza

\vfill
\bTABLE
    \bTR\bTD t.:\eTD\bTD Joseph Mohr (1818~r.)\eTD\eTR
    \bTR\bTD tłum.:\eTD\bTD Piotr Maszyński (ok. 1930~r.)\eTD\eTR
    \bTR\bTD m.:\eTD\bTD Franz Xaver Gruber (1818~r.)\eTD\eTR
\eTABLE


\startsection[title={Dlaczego dzisiaj wśród nocy dnieje}] %{{{1
% http://bibliotekapiosenki.pl/Dlaczego_dzisiaj_wsrod_nocy_dnieje
\author{Kaszycki, Jan}
\composer{Markiewicz}

\startstanza
Dlaczego dzisiaj wśród nocy dnieje
i~jako słońce niebo jaśnieje?\stopstanza
\startrefrain
    Chrystus, Chrystus nam się narodził
    aby nas od piekła oswobodził!\stoprefrain

\startstanza
Dlaczego dzisiaj, Boży aniele
ogłaszasz ludziom wielkie wesele?\stopstanza
\startrefrain
    Chrystus, Chrystus nam się narodził
    aby nas od piekła oswobodził!\stoprefrain

\startstanza
Czemuż pasterze do szopy śpieszą
i~podarunki ze sobą niosą?\stopstanza
\startrefrain
    Chrystus, Chrystus nam się narodził
    aby nas od piekła oswobodził!\stoprefrain

\startstanza
Czemuż wół, osioł społem klękają,
małej Dziecinie pokłon oddają?\stopstanza
\startrefrain
    Chrystus, Chrystus nam się narodził
    aby nas od piekła oswobodził!\stoprefrain

\startstanza
Dlaczego gwiazda nad podziw świeci
i~przed Królami tak szybko leci?\stopstanza
\startrefrain
    Chrystus, Chrystus nam się narodził
    aby nas od piekła oswobodził!\stoprefrain

\vfill
\bTABLE
    \bTR\bTD t.:\eTD\bTD Jan Kaszycki (1911~r.)\eTD\eTR % "Kantyczki z nutami"?
    \bTR\bTD m.:\eTD\bTD Markiewicz\eTD\eTR
\eTABLE


\startsection[title={Do szopy hej pasterze}] %{{{1
% http://bibliotekapiosenki.pl/Do_szopy_hej_pasterze
\author{Gwoździowski, J. A.}

\startstanza
Do szopy, hej pasterze,
do szopy, bo tam cud!
Syn Boży w~żłobie leży,
by zbawić ludzki ród.\stopstanza
\startrefrain
    Śpiewajcie Aniołowie,
    pasterze, grajcie Mu.
    Kłaniajcie się Królowie,
    nie budźcie Go ze snu.\stoprefrain

\startstanza
Padnijmy na kolana,
to Dziecię to nasz Bóg.
Witajmy swego Pana,
wdzięczności złóżmy dług.\stopstanza
\startrefrain
    Śpiewajcie...\stoprefrain

\startstanza
O~Boże niepojęty, kto
pojmie miłość Twą?
Na sianie wśród bydlęty
masz tron i~służbę swą.\stopstanza
\startrefrain
    Śpiewajcie...\stoprefrain

\startstanza
Bóg, Stwórca wiecznej chwały,
Bóg godzien wszelkiej czci,
patrz, w~szopie tej zbutwiałej,
jak słodko On tam śpi.\stopstanza
\startrefrain
    Śpiewajcie...\stoprefrain

\vfill
\bTABLE
    \bTR\bTD t.~i~m.:\eTD\bTD \quote{Największa kantyczka}, J.~A.~Gwoździowski (1938~r.)\eTD\eTR
\eTABLE



\startsection[title={Dzisiaj w~Betlejem}] %{{{1
% http://bibliotekapiosenki.pl/Dzisiaj_w_Betlejem
\author{Siedlecki, ks. Jan, CM}

\startstanza
Dzisiaj w~Betlejem, dzisiaj w~Betlejem wesoła nowina,
że Panna czysta, że Panna czysta porodziła Syna.\stopstanza
\startrefrain
    Chrystus się rodzi, nas oswobodzi,
    anieli grają, króle witają,
    pasterze śpiewają, bydlęta klękają,
    cuda, cuda ogłaszają.\stoprefrain

\startstanza
Maryja Panna, Maryja Panna Dzieciątko piastuje
i~Józef stary, i~Józef stary Ono pielęgnuje.\stopstanza
\startrefrain
    Chrystus się rodzi...\stoprefrain

\startstanza
Choć w~stajeneczce, choć w~stajeneczce Panna Syna rodzi,
przecież On wkrótce, przecież On wkrótce ludzi oswobodzi.\stopstanza
\startrefrain
    Chrystus się rodzi...\stoprefrain

\startstanza
I~Trzej Królowie, i~Trzej Królowie od wschodu przybyli
i~dary Panu, i~dary Panu kosztowne złożyli.\stopstanza
\startrefrain
    Chrystus się rodzi...\stoprefrain

\startstanza
Pójdźmy też i~my, pójdźmy też i~my przywitać Jezusa,
Króla nad królmi, Króla nad królmi uwielbić Chrystusa.\stopstanza
\startrefrain
    Chrystus się rodzi...\stoprefrain

\startstanza
Bądźże pochwalon, bądźże pochwalon dziś nasz wieczny Panie,
któryś złożony, któryś złożony na zielonem sianie.\stopstanza
\startrefrain
    Chrystus się rodzi...\stoprefrain

\startstanza
Bądź pozdrowiony, bądź pozdrowiony Boże nieskończony.
Sławimy Ciebie, sławimy Ciebie Jezu niezmierzony.\stopstanza
\startrefrain
    Chrystus się rodzi...\stoprefrain


\vfill
\bTABLE
    \bTR\bTD t.~i~m.:\eTD\bTD \quote{Śpiewniczek}, ks. Jan Siedlecki CM (1878~r.)\eTD\eTR
\eTABLE


% Filler: choinka {{{1
\page[yes]
\strut
\vskip 0pt plus 3fill
\startalignment[middle]\dontleavehmode
\externalfigure[rysunki/choinka][align=middle,width=0.75\textwidth,]%
\stopalignment
\vskip 0pt plus 2fill


\startsection[title={Gdy się Chrystus rodzi}] %{{{1
% http://bibliotekapiosenki.pl/Gdy_sie_Chrystus_rodzi
\author{Mioduszewski, ks. Michał, CM+„Pastorałki i kolędy”}

\startstanza
Gdy się Chrystus rodzi i~na świat przychodzi
ciemna noc w~jasnościach promienistych brodzi.
Aniołowie się radują,
pod niebiosy wyśpiewują:
Gloria, gloria, gloria, in excelsis Deo!\stopstanza

\startstanza
Mówią do pasterzy, którzy trzód swych strzegli
aby do Betlejem czem prędzej pobiegli.
Bo się narodził Zbawiciel,
wszego świata Odkupiciel;
Gloria, gloria, gloria, in excelsis Deo!\stopstanza

\startstanza
O~niebieskie duchy i~posłowie nieba!
Powiedzcież wyraźniej, co nam czynić trzeba.
Bo my nic nie pojmujemy,
ledwo od strachu żyjemy;
Gloria, gloria, gloria, in excelsis Deo!\stopstanza

\startstanza
Idźcież do Betlejem, gdzie Dziecię zrodzone,
W~pieluszki powite, w~żłobie położone.
Oddajcie Mu pokłon Boski,
On osłodzi wasze troski:
Gloria, gloria, gloria, in excelsis Deo!\stopstanza

\startstanza
A~gdy pastuszkowie wszystko zrozumieli
zaraz do Betlejem śpieszno pobieżeli
i~zupełnie tak zastali
jak Anieli im zeznali.
Gloria, gloria, gloria, in excelsis Deo!\stopstanza

\page[yes]

\startstanza
A~stanąwszy w~miejscu pełni zadumienia,
iż się Bóg tak zniżył do swego stworzenia,
padli przed Nim na kolana
i~uczcili swego Pana:
Gloria, gloria, gloria, in excelsis Deo!\stopstanza

\startstanza
Nareszcie, gdy pokłon Panu już oddali
z~wielką wesołością do swych trzód wracali,
że się stali być godnymi
Boga widzieć na tej ziemi.
Gloria, gloria, gloria, in excelsis Deo!\stopstanza

%\vfill
\fillersmall

\bTABLE
    \bTR\bTD t.~i~m.:\eTD\bTD \quote{Pastorałki i~kolędy}, ks. Michał Mioduszewski CM (1843 r.)\eTD\eTR
    \bTR\bTD m.:\eTD\bTD być może starofrancuska\eTD\eTR
\eTABLE


\startsection[title={Gdy śliczna Panna}] %{{{1
% http://bibliotekapiosenki.pl/Gdy_sliczna_Panna
\author{karmelitanki}
\author{Mioduszewski, ks. Michał, CM+„Pastorałki i kolędy”}

\startstanza
Gdy śliczna Panna Syna kołysała
z~wielkim weselem tak Jemu śpiewała:\stopstanza
\startrefrain
    Lili lili laj, moje dzieciąteczko,
    lili lili laj, śliczne paniąteczko.\stoprefrain

\startstanza
Wszystko stworzenie śpiewaj Panu swemu,
pomóż radości wielkiej sercu memu.\stopstanza
\startrefrain
    Lili lili laj, wielki królewicu,
    lili lili laj, niebieski dziedzicu.\stoprefrain

\startstanza
Sypcie się z~nieba liczni aniołowie,
śpiewajcie Panu niebiescy duchowie.\stopstanza
\startrefrain
    Lili lili laj, mój wonny kwiateczku,
    lili lili laj, w~ubogim żłóbeczku.\stoprefrain

\startstanza
Cicho wietrzyku, cicho południowy,
cicho powiewiaj, niech śpi Panicz nowy:\stopstanza
\startrefrain
    Lili lili laj, mój wdzięczny Synaczku,
    lili lili laj, miluchny robaczku.\stoprefrain

\vfill
\bTABLE
    \bTR\bTD t.:\eTD\bTD w~kantyczkach karmelitańskich (XVIII~w.)\eTD\eTR
    \bTR\bTD m.:\eTD\bTD \quote{Pastorałki i~kolędy}, ks. Michał Mioduszewski CM (1843~r.)\eTD\eTR
\eTABLE

\startsection[title={Gore gwiazda Jezusowi}] %{{{1
% http://bibliotekapiosenki.pl/Gore_gwiazda_Jezusowi
\author{karmelitanki}
\author{Mioduszewski, ks. Michał, CM+„Pastorałki i kolędy”}

\startstanza
Gore gwiazda Jezusowi w~obłoku, w~obłoku,
Józef z~Panną asystują przy boku, przy boku.\stopstanza
\startrefrain
    Hojże ino dyny dyna,
    narodził się Bóg-Dziecina
    w~Betlejem, Betlejem.\stoprefrain

\startstanza
Wół i~osioł w~parze służą przy żłobie, przy żłobie,
huczą, buczą delikatnej osobie, osobie.\stopstanza
\startrefrain
    Hojże ino...\stoprefrain

\startstanza
Pastuszkowie z~podarunki przybiegli, przybiegli,
wkoło szopę o~północy obiegli, obiegli.\stopstanza
\startrefrain
    Hojże ino...\stoprefrain

\startstanza
Anioł Pański, sam ogłosił te dziwy, te dziwy,
których oni nie słyszeli jak żywi, jak żywi.\stopstanza
\startrefrain
    Hojże ino...\stoprefrain

\startstanza
Anioł Pański kuranciki wycina, wycina,
skąd pociecha dla człowieka jedyna, jedyna.\stopstanza
\startrefrain
    Hojże ino...\stoprefrain

\startstanza
Już Maryja Jezuleńka powiła, powiła,
nam wesele i~pociecha stąd miła, stąd miła.\stopstanza
\startrefrain
    Hojże ino...\stoprefrain

\vfill
\bTABLE
    \bTR\bTD t.:\eTD\bTD w~kantyczkach karmelitańskich (XVIII~w.)\eTD\eTR
    \bTR\bTD m.:\eTD\bTD \quote{Pastorałki i~kolędy}, ks. Michał Mioduszewski CM (1843~r.)\eTD\eTR
\eTABLE

\startsection[title={Gwiazdka na wschodzie}] %{{{1
\author{Smolarkiewicz, ks. Władysław}

\startstanza
Gwiazdka na wschodzie tak jasno lśni
i~mruga na nas, byśmy tam szli,
gdzie jako zorza Dziecina Boża
na świat cały rzuca blask.\stopstanza

\startstanza
Łączmy głosy w~jeden ton
i~serc naszych nieśmy plon,
gdzie jako zorza Dziecina Boża
na świat cały rzuca blask.\stopstanza

\startstanza
Wyciąga rączki do grzesznych sług,
w~maleńkim ciele potężny Bóg;
berła i~łany, nędza, łachmany
otaczają Jego żłób.\stopstanza

\vfill
\bTABLE
    \bTR\bTD t.~i~m.:\eTD\bTD „Pieśni Bożego Narodzenia. Kolędy kościelne i~domowe\eTD\eTR
    \bTR\bTD\eTD\bTD oraz doroczne pieśni kościelne.”, ks. Władysław Smolarkiewicz\eTD\eTR
\eTABLE


\startsection[title={Hej, w~dzień Narodzenia}] %{{{1
% http://bibliotekapiosenki.pl/Hej_w_dzien_Narodzenia
\author{Staniątki}
\author{Mioduszewski, ks. Michał, CM+„Pastorałki i kolędy”}

\startstanza
Hej, w~dzień narodzenia Syna Jedynego
Ojca Przedwiecznego, Boga prawdziwego,
wesoło śpiewajmy, chwałę Bogu dajmy.\stopstanza
\startrefrain
    Hej kolęda, kolęda!\stoprefrain

\startstanza
Panna porodziła niebieskie Dzieciątko,
w~żłobie położyła małe pacholątko;
pasterze śpiewają, na multankach grają.\stopstanza
\startrefrain
    Hej kolęda, kolęda!\stoprefrain

\startstanza
Skoro pastuszkowie o~tym usłyszeli,
zaraz do Betlejem czem prędzej bieżeli,
witając Dzieciątko, małe pacholątko.\stopstanza
\startrefrain
    Hej kolęda, kolęda!\stoprefrain

\vfill
\bTABLE
    \bTR\bTD t.:\eTD\bTD w~kancjonałach benedyktynek staniąteckich (1707~r.)\eTD\eTR
    \bTR\bTD m.:\eTD\bTD \quote{Pastorałki i~kolędy}, ks. Michał Mioduszewski CM (1843~r.)\eTD\eTR
\eTABLE


\startsection[title={Jezusa narodzonego}] %{{{1
\author{Keller, Szymon}
\composer{Walczyński, ks. Franciszek}

\startstanza
Jezusa narodzonego wszyscy witajmy,
Jemu po kolędzie dary wzajem oddajmy.\stopstanza
\startrefrain
    Oddajmy wesoło, skłaniajmy swe czoło,
    Panu naszemu.\stoprefrain

\startstanza
Oddajmy za złoto wiarę, czyniąc wyznanie,
że to Bóg i~Człowiek prawy leży na sianie.\stopstanza
\startrefrain
    Oddajmy wesoło...\stoprefrain

\startstanza
Oddajmy też za kadzidło Panu nadzieję,
że Go będziem widzieć w~niebie, mówiąc to śmiele.\stopstanza
\startrefrain
    Oddajmy wesoło...\stoprefrain

\startstanza
Oddajmy za mirrę miłość na dowód tego,
że Go nad wszystko kochamy z~serca całego.\stopstanza
\startrefrain
    Oddajmy wesoło...\stoprefrain

\startstanza
Przyjmij, Jezu, po kolędzie te nasze dary,
przyjmij serca, dusze nasze za swe ofiary.\stopstanza
\startrefrain
    Byśmy kiedyś w~niebie posiąść mogli Ciebie,
    posiąść mogli Ciebie na wieki wieków.\stoprefrain

\vfill
\bTABLE
    \bTR\bTD t.:\eTD\bTD Szymon Keller (1868~r.)\eTD\eTR
    \bTR\bTD m.:\eTD\bTD ks. Franciszek Walczyński (1884~r.)\eTD\eTR
\eTABLE


\startsection[title={Jezus malusieńki}] %{{{1
% http://bibliotekapiosenki.pl/Jezus_malusienki_%28koleda%29
\author{karmelitanki}
\author{Staniątki}
\author{Mioduszewski, ks. Michał, CM+„Pastorałki i kolędy”}

\startstanza
Jezus malusieńki leży wśród stajenki,
płacze z~zimna, nie dała mu Matusia sukienki.\stopstanza

\startstanza
Bo uboga była, rąbek z~głowy zdjęła,
w~który Dziecię owinąwszy siankiem Je okryła.\stopstanza

\startstanza
Nie ma kolebeczki ani poduszeczki,
we żłobie Mu położyła siana pod główeczki.\stopstanza

\startstanza
Dziecina się kwili, Matuleńka lili,
w~nóżki zimno, żłobek twardy, stajenka się chyli.\stopstanza

\startstanza
Matusia truchleje, serdeczne łzy leje:
o~mój Synu, wola Twoja, nie moja się dzieje.\stopstanza

\startstanza
Przestań płakać, proszę, bo żalu nie zniosę,
dosyć go mam z~męki Twojej, którą w~sercu noszę.\stopstanza

\startstanza
Józefie stareńki, daj z~ogniem fajerki
grzać Dziecinę, sam co prędzej podpieraj stajenki.\stopstanza

\startstanza
Pokłon oddawajmy, Bogiem go wyznajmy,
to Dzieciątko ubożuchne ludziom ogłaszajmy.\stopstanza

\vfill
\bTABLE
    \bTR\bTD t.:\eTD\bTD w~kancjonale Gąsiorowskiej ze Staniątek (1754~r.)\eTD\eTR
    \bTR\bTD\eTD\bTD i~w~kantyczkach karmelitanek z~Krakowa (XVIII~w.)\eTD\eTR
    \bTR\bTD m.:\eTD\bTD \quote{Pastorałki i~kolędy}, ks. Michał Mioduszewski CM (1843~r.)\eTD\eTR
\eTABLE


\startsection[title={Lulajże, Jezuniu}] %{{{1
% http://bibliotekapiosenki.pl/Lulajze_Jezuniu_%28koleda%29
\author{Staniątki}
\author{Mioduszewski, ks. Michał, CM+„Pastorałki i kolędy”}

\startstanza
Lulajże, Jezuniu, moja perełko,
lulaj ulubione me pieścidełko.\stopstanza
\startrefrain
    Lulajże, Jezuniu, lulajże lulaj!
    A~ty Go, Matulu, w~płaczu utulaj\stoprefrain

\startstanza
Zamknijże znużone płaczem powieczki,
utulże zemdlone łkaniem usteczki.\stopstanza
\startrefrain
    Lulajże, Jezuniu...\stoprefrain

\startstanza
Lulajże, piękniuchny nasz aniołeczku,
lulajże, wdzięczniuchny świata kwiateczku.\stopstanza
\startrefrain
    Lulajże, Jezuniu...\stoprefrain

\fillersmall
\bTABLE
    \bTR\bTD t.:\eTD\bTD (II~poł.~XVIIw.); w~kancjonale Gąsiorowskiej ze Staniątek (1754~r.)\eTD\eTR
    \bTR\bTD\eTD\bTD i~w~kancjonałach franciszkańskich (XVIII~w.)\eTD\eTR
    \bTR\bTD m.:\eTD\bTD \quote{Pastorałki i~kolędy}, ks. Michał Mioduszewski CM (1843~r.)\eTD\eTR
\eTABLE


% Filler: świeczka {{{1
\page[yes]
\strut
\vskip 0pt plus 2fill%
\hskip 0pt plus 3fill%
\externalfigure[rysunki/świeczka][width=0.5\textwidth]%
\hskip 0pt plus 2fill\strut%
\vskip 0pt plus 1fill%


\startsection[title={Mędrcy świata}] %{{{1
% http://bibliotekapiosenki.pl/Medrcy_swiata
\author{Bortkiewicz, Stefan}
\composer{Odelgiewicz, ks.~Zygmunt}

\startstanza
Mędrcy świata, monarchowie,
gdzie śpiesznie dążycie?
Powiedzcież nam, Trzej Królowie,
chcecie widzieć Dziecię?
Ono w~żłobie nie ma tronu
i~berła nie dzierży,
a~proroctwo jego zgonu
już się w~świecie szerzy.\stopstanza

\startstanza
Mędrcy świata, złość okrutna
Dziecię prześladuje,
wieść to straszna, wieść to smutna,
Herod spisek knuje.
Nic monarchów nie odstrasza,
do Betlejem spieszą,
gwiazda Zbawcę im ogłasza,
nadzieją się cieszą.\stopstanza

\startstanza
Przed Maryją stają społem,
niosą Panu dary.
Przed Jezusem biją czołem,
składają ofiary.
Trzykroć szczęśliwi królowie,
któż wam nie zazdrości?
Cóż my damy, kto nam powie,
pałając z~miłości?\stopstanza

\page[yes]

\startstanza
Oto jak każą kapłani,
damy dar troisty:
modły, pracę niosąc w~dani
i~żar serca czysty.
To kadzidło, mirrę, złoto,
niesiem, Jezu szczerze.
Co dajemy Ci z~ochotą
od nas przyjm w~ofierze.\stopstanza

%\vfill
\fillersmall
\bTABLE
    \bTR\bTD t.:\eTD\bTD Stefan Bortkiewicz (ok.~1870~r.)\eTD\eTR
    \bTR\bTD m.:\eTD\bTD ks. Zygmunt Odelgiewicz (XIX~w.)\eTD\eTR
\eTABLE


\startsection[title={Mizerna cicha}] %{{{1
% http://bibliotekapiosenki.pl/Mizerna_cicha_stajenka
\author{Lenartowicz, Teofil}
\composer{Wrzeciono, Jakub}
\composer{Gall, Jan Karol}

\startstanza
Mizerna, cicha, stajenka licha,
pełna niebieskiej chwały;
oto leżący, przed nami śpiący
w~promieniach Jezus mały.\stopstanza

\startstanza
Nad nim Anieli w~locie stanęli
i~pochyleni klęczą
z~włosy złotymi, z~skrzydły białymi
pod malowaną tęczą.\stopstanza

\startstanza
Wielkie zdziwienie, wszelkie stworzenie,
cały świat orzeźwiony;
Mądrość mądrości, Światłość światłości,
Bóg-człowiek tu wcielony!\stopstanza

\startstanza
I~oto mnodzy, ludzie ubodzy
radzi oglądać Pana;
pełni natchnienia, pełni zdziwienia
upadli na kolana.\stopstanza

\startstanza
Długo czekali, długo wzdychali,
aż niebo rozgorzało,
piekło zawarte, niebo otwarte,
Słowo się Ciałem stało.\stopstanza

%\page[yes]
\startstanza
Hej, ludzie prości, Bóg z~nami gości,
skończony czas niedoli.
On daje siebie, chwała na niebie,
Mir ludziom dobrej woli.\stopstanza

\startstanza
Radość na ziemi, bo nad wszystkimi
roztacza blask rumiany.
Przepaść rozwarta, upadek czarta,
Zstępuje Pan nad pany.\stopstanza

%\fillersmall
\vfill
\bTABLE
    \bTR\bTD t.:\eTD\bTD Teofil Lenartowicz (1849~r.)\eTD\eTR
    \bTR\bTD m.:\eTD\bTD Jakub Wrzeciono\eTD\eTR
    \bTR\bTD m.:\eTD\bTD Jan Karol Gall\eTD\eTR
\eTABLE


\startsection[title={Nowy Rok bieży}] %{{{1
% http://bibliotekapiosenki.pl/Nowy_rok_biezy
\author{Szlichtyn, Jan}

\startstanza
Nowy rok bieży, w~jasełkach leży,
a~kto, kto?
Dzieciątko małe, dajcie mu chwałę,
na ziemi.\stopstanza

\startstanza
Leży Dzieciątko jako jagniątko,
a~gdzie, gdzie?
W~Betlejem mieście, tam się pospieszcie,
znajdziecie.\stopstanza

\startstanza
Jak Go poznamy, gdy Go nie znamy,
Jezusa?
Podło uwity, nie w~aksamity,
ubogo.\stopstanza

\startstanza
Wół, osioł ziewa, parą zagrzewa,
a~jakoż?
Klęcząc, padając, chwałę oddając
przy żłobie.\stopstanza

\startstanza
Anieli grają, wdzięcznie śpiewają,
a~co, co?
Niech chwała będzie, zawsze i~wszędzie
Dzieciątku.\stopstanza

%\page[yes]
\startstanza
Królowie jadą z~wielką gromadą,
a~skąd, skąd?
Od wschodu słońca, szukają końca,
zbawienia.\stopstanza

\startstanza
Skarb otwierają, dary dawają,
a~komu?
Wielcy Królowie, możni Panowie,
Dzieciątku.\stopstanza

%\fillersmall
\vfill

\bTABLE
    \bTR\bTD t.:\eTD\bTD w~jezuickich \quote{Pieśniach nabożnych} (1745~r.)\eTD\eTR
    \bTR\bTD\eTD\bTD i~w~rozszerzeniu \quote{Kantyczek pieśni nabożnych} Jana Szlichtyna (1785~r.)\eTD\eTR
    \bTR\bTD m.:\eTD\bTD autor nieznany\eTD\eTR
\eTABLE


\startsection[title={O~gwiazdo betlejemska}] %{{{1
\author{Odelgiewicz, ks.~Zygmunt}
\author{Orszulik, Antoni}
\author{Siedlecki, ks. Jan, CM}

\startstanza
O~gwiazdo Betlejemska, zaświeć na niebie mym.
% w oryginale "Tak szukam" XXX
Ja szukam Cię wśród nocy, tęsknię za światłem Twym.
Zaprowadź do stajenki, leży tam Boży Syn,
Bóg-człowiek z~Panny świętej, dany na okup win.\stopstanza

\startstanza
O~nie masz Go już w~szopce, nie masz Go w~żłóbku tam,
więc gdzie pójdziemy Chryste? Gdzie się ukryłeś nam?
Pójdziemy przed ołtarze, wzniecić miłości żar
i~hołd Ci niski oddać: to jest nasz wszystek dar.\stopstanza

\startstanza
Ja nie wiem, o~mój Panie, któryś miał w~żłobie tron,
czy dusza moja biedna milsza Ci jest, niż on?
Ulituj się nade mną, błagać Cię kornie śmiem,
gdyś stajnią nie pogardził, nie gardź i~sercem mym.\stopstanza

\vfill
\bTABLE
    \bTR\bTD t.:\eTD\bTD prawdopodobnie ks.~Zygmunt Odelgiewicz w~oprac. Antoniego Orszulika\eTD\eTR
    \bTR\bTD m.:\eTD\bTD \quote{Śpiewnik kościelny} ks.~Jan Siedlecki CM (1928~r.)\eTD\eTR
\eTABLE


% Filler: gwiazdka {{{1
\page[yes]\strut
\vskip 0pt plus 3fill
\startalignment[middle]\dontleavehmode
\externalfigure[rysunki/gwiazdka][align=middle,width=0.75\textwidth,]%
\stopalignment
\vskip 0pt plus 2fill

 
\startsection[title={Pasterze mili, coście widzieli}] %{{{1
% http://bibliotekapiosenki.pl/Pasterze_mili
\author{Szlichtyn, Jan}
\author{Staniątki}

\startstanza
Pasterze mili, coście widzieli?
Widzieliśmy maleńkiego Jezusa narodzonego,
Syna Bożego.\stopstanza

\startstanza
Co za pałac miał, gdzie gospodą stał?
Szopa bydłu przyzwoita i~to jeszcze źle pokryta
pałacem była.\stopstanza

\startstanza
Jakie łóżeczko, miał Paniąteczko?
Marmur twardy, żłób kamienny, na tem depozyt zbawienny
spoczywał łożu.\stopstanza

\startstanza
Co za obicie miało to Dziecię?
Wisząc spod strzech pajęczyna, Boga i~Maryi syna
obiciem była.\stopstanza

\startstanza
W~jakiej odzieży Pan nieba leży?
Za purpurę, perły drogie ustroiła Go w~ubogie
pieluszki nędza.\stopstanza

\startstanza
Czyli w~wygodach, czy spał w~swobodach?
Na barłogu, ostrem sianie, delikatne spało Panię
a~nie w~łabędziach.\stopstanza

\startstanza
Kto asystował? Kto Go pilnował?
Wół i~osieł przyklękali, parą swą Go zagrzewali
dworzanie Jego.\stopstanza

\startstanza
Jakie kapele nuciły trele?
Aniołowie Mu śpiewali, my na dudkach przygrywali
skoczno wesoło.\stopstanza

\page[yes]
\startstanza
Kto więcej śpieszył, by Dziecię cieszył?
Józef stary z~Panieneczką za melodyjną piosneczką
Dziecię cieszyli.\stopstanza

\startstanza
Jakieście dary dali, ofiary?
Sercaśmy własne oddali a~odchodząc poklękali,
czołem Mu bili.\stopstanza

\fillersmall
\bTABLE
    \bTR\bTD t.:\eTD\bTD w~kantyczkach karmelitańskich (XVIII~w.)\eTD\eTR
    \bTR\bTD\eTD\bTD i~w~\quote{Kantyczkach pieśni nabożnych} Jana Szlichtyna (1785~r.)\eTD\eTR
    \bTR\bTD m.:\eTD\bTD w~kancjonałach~benedyktynek~staniąteckich\eTD\eTR
\eTABLE


\startsection[title={Pójdźmy wszyscy do stajenki}] %{{{1
% http://bibliotekapiosenki.pl/Pojdzmy_wszyscy_do_stajenki
\author{Mioduszewski, ks. Michał, CM+„Śpiewnik kościelny”}
\author{Siedlecki, ks. Jan, CM}
\author{Kraków}

\startstanza
Pójdźmy wszyscy do stajenki,
do Jezusa i~Panienki,
powitajmy Maleńkiego
i~Maryję Matkę Jego.\stopstanza

\startstanza
Witaj, Jezu ukochany,
od patriarchów czekany,
od proroków ogłoszony,
od narodów upragniony.\stopstanza

\startstanza
Witaj, Dzieciąteczko w~żłobie,
wyznajemy Boga w~Tobie,
coś się narodził tej nocy,
byś nas wyrwał z~czarta mocy.\stopstanza

\startstanza
Witaj Jezu nam zjawiony,
witaj dwakroć narodzony,
raz z~Ojca przed wieków wiekiem
a~teraz z~matki człowiekiem.\stopstanza

\startstanza
Któż to słyszał takie dziwy?
Tyś człowiek i~Bóg prawdziwy,
Ty łączysz w~boskiej osobie
dwie natury różne sobie.\stopstanza

%\page[yes]
\startstanza
O~szczęśliwi pastuszkowie!
Któż radość waszą wypowie?
Czego Ojcowie żądali
wyście pierwsi oglądali.\stopstanza

\startstanza
Święta Panno, twa przyczyna
niech nam wyjedna u~Syna,
by to jego narodzenie
zapewniło nam zbawienie.\stopstanza

%\fillersmall
\vfill
\bTABLE
    \bTR\bTD t.:\eTD\bTD \quote{Dodatek do Śpiewnika kościelnego} ks. M. Mioduszewskiego CM (1842~r.)\eTD\eTR
    \bTR\bTD m.:\eTD\bTD z~Krakowa; \quote{Śpiewniczek}, ks. Jan Siedlecki (1878~r.)\eTD\eTR
\eTABLE


\startsection[title={Przybieżeli do Betlejem\\(31.~symfonia anielska)}, %{{{1
    bookmark={Przybieżeli do Betlejem}]
% http://bibliotekapiosenki.pl/Pojdzmy_wszyscy_do_stajenki
\author{Żabczyc, Jan}
\author{Mioduszewski, ks. Michał, CM+„Pastorałki i kolędy”}

\startstanza
Przybieżeli do Betlejem pasterze,
grają skocznie Dzieciąteczku na lirze.\stopstanza
\startrefrain
    Chwała na wysokości, chwała na wysokości,
    a~pokój na ziemi.\stoprefrain

\startstanza
Oddawali swe ukłony w~pokorze
Tobie z~serca ochotnego, o~Boże!\stopstanza
\startrefrain
     Chwała na wysokości...\stoprefrain

\startstanza
Anioł Pański sam ogłosił te dziwy,
których oni nie słyszeli, jak żywi.\stopstanza
\startrefrain
    Chwała na wysokości...\stoprefrain

\startstanza
Dziwili się napowietrznej muzyce
i~myśleli, co to będzie za Dziecię?\stopstanza
\startrefrain
    Chwała na wysokości...\stoprefrain

\startstanza
Oto mu się wół i~osioł kłaniają,
trzej królowie podarunki oddają.\stopstanza
\startrefrain
    Chwała na wysokości...\stoprefrain

\startstanza
I~anieli gromadami pilnują,
Panna czysta wraz z~Józefem piastują.\stopstanza
\startrefrain
    Chwała na wysokości...\stoprefrain

%\page[yes]
\startstanza
Poznali Go Mesyjaszem być prawym,
narodzonym dzisiaj Panem łaskawym.\stopstanza
\startrefrain
    Chwała na wysokości...\stoprefrain

\startstanza
My go także Bogiem, Zbawcą już znamy
i~z~całego serca wszyscy kochamy.\stopstanza
\startrefrain
    Chwała na wysokości...\stoprefrain

%\fillersmall

\vfill
\bTABLE
    \bTR\bTD t.:\eTD\bTD \quote{Symfonie anielskie}, Jan Żabczyc (1630~r.)\eTD\eTR
    \bTR\bTD m.:\eTD\bTD \quote{Dodatek do Pastorałek i~kolęd}, ks. Michał Mioduszewski CM (1853~r.)\eTD\eTR
\eTABLE


\startsection[title={Triumfy Króla niebieskiego}] %{{{1
% http://bibliotekapiosenki.pl/Triumfy_Krola_Niebieskiego
\author{Staniątki}
\author{Siedlecki, ks. Jan, CM}

\startstanza
Triumfy Króla niebieskiego
zstąpiły z~nieba wysokiego.
Pobudziły pasterzów,
dobytku swego stróżów,
śpiewaniem, śpiewaniem, śpiewaniem.\stopstanza

\startstanza
Chwała bądź Bogu w~wysokości
a~ludziom pokój na niskości.
Narodził się Zbawiciel,
dusz ludzkich Odkupiciel,
na ziemi, na ziemi, na ziemi.\stopstanza

\startstanza
Zrodziła Maryja Dziewica
wiecznego Boga bez rodzica.
By nas z~piekła wybawił
a~w~niebieskich postawił
pałacach, pałacach, pałacach.\stopstanza

\startstanza
Pasterze w~podziwieniu stają,
triumfu przyczynę badają.
Co się nowego dzieje,
że tak światłość jaśnieje,
nie wiedząc, nie wiedząc, nie wiedząc.\stopstanza

\startstanza
Że to Bóg, gdy się dowiedzieli,
swej trzody w~polu odbieżeli
spiesząc na powitanie
do Betlejemskiej stajni
Dzieciątka, Dzieciątka, Dzieciątka.\stopstanza

\vfill
\bTABLE
    \bTR\bTD t.:\eTD\bTD w~kancjonałach benedyktynek staniąteckich (XVII~w.)\eTD\eTR
    \bTR\bTD m.:\eTD\bTD \quote{Śpiewniczek}, ks. Jan Siedlecki CM (1879~r.)\eTD\eTR
\eTABLE


% Filler: dzwonek {{{1
\page[yes]\strut
\vskip 0pt plus 3fill
\startalignment[middle]\dontleavehmode
\externalfigure[rysunki/dzwonek][align=middle,width=0.75\textwidth,]%
\stopalignment
\vskip 0pt plus 2fill


\startsection[title={W~dzień Bożego Narodzenia}] %{{{1
% http://bibliotekapiosenki.pl/W_dzien_Bozego_Narodzenia_%28koleda%29
\author{Staniątki}
\author{Szlichtyn, Jan}
\composer{Szumlański, Feliks}

\startstanza
W~dzień Bożego Narodzenia
radość wszelkiego stworzenia,
ptaszki w~górę podlatują,
Jezusowi wyśpiewują,
wyśpiewują.\stopstanza

\startstanza
Słowik zaczyna dyszkantem,
szczygieł mu wtóruje altem,
szpak tenorem krzyknie czasem,
a~gołąbek gruchnie basem,
gruchnie basem.\stopstanza

\startstanza
Wróbel, ptaszek nieboraczek,
uziąbłszy śpiewa jak żaczek,
Dziw, dziw, dziw, dziw, dziw nad dziwy,
narodził się Bóg prawdziwy,
Bóg prawdziwy.\stopstanza

\startstanza
A~mazurek ze swym synem
tak świergocą za kominem;
cierp, cierp, cierp, cierp, miły Panie
póki mróz ten nie ustanie, % tak się śpiewa u nas w rodzinie
nie ustanie.\stopstanza

\startstanza
I~żurawie w~swoje nosy
wykrzykują pod niebiosy;
czajka w~górę podlatuje,
chwałę Bogu wyśpiewuje,
wyśpiewuje.\stopstanza

\page[yes]
\startstanza
Sroka wlazła na jedlinę
odarła sobie łysinę
i~choć gołe świeci czoło,
śpiewa jednak dość wesoło,
dość wesoło.\stopstanza

\startstanza
Kur na grzędzie krzyczy wszędzie:
wstańcie, ludzie, bo dzień będzie!
Do Betlejem pospieszajcie,
Boga w~ciele oglądajcie,
oglądajcie.\stopstanza

\fillersmall% FIXME ptaszek

\bTABLE
    \bTR\bTD t.:\eTD\bTD w~kantyczkach karmelitańskich (XVIII~w.)\eTD\eTR
    \bTR\bTD\eTD\bTD i~\quote{Kantyczki pieśni nabożnych} Jana Szlichtyna (1767~r.)\eTD\eTR
    \bTR\bTD m.:\eTD\bTD Feliks Szumlański (XIX~w.)\eTD\eTR
\eTABLE


\startsection[title={Wesołą nowinę}] %{{{1
% http://bibliotekapiosenki.pl/Wesola_nowine
\author{Siedlecki, ks. Jan, CM}
\composer{Wygrzewalski, Józef}

\startstanza
Wesołą nowinę, bracia, słuchajcie,
niebieską Dziecinę ze mną witajcie.\stopstanza
\startrefrain
    Jak miła ta nowina!
    Mów, gdzie jest ta Dziecina?
    Byśmy tam pobieżeli i~ujrzeli.\stoprefrain

\startstanza
Bogu chwałę wznoszą na wysokości,
pokój ludziom głoszą duchy światłości.\stopstanza
\startrefrain
    Jak miła...\stoprefrain

\startstanza
Panna nam powiła Boskie Dzieciątko,
pokłonem uczciła to niemowlątko.\stopstanza
\startrefrain
    Jak miła...\stoprefrain

\startstanza
Którego zrodziła, Bogiem uznała,
i~Panną, jak była, Panną została.\stopstanza
\startrefrain
    Jak miła...\stoprefrain

\startstanza
Królowie na wschodzie już to poznali
i~w~judzkim narodzie szukać jechali.\stopstanza
\startrefrain
    Jak miła...\stoprefrain

\startstanza
Gwiazda najśliczniejsza ich oświeciła,
do szopy w~Betlejem zaprowadziła.\stopstanza
\startrefrain
    Jak miła...\stoprefrain

\startstanza
Znaleźli to Dziecię i~Matkę Jego.
Tam idźcie, znajdziecie Syna Bożego!\stopstanza
\startrefrain
    Jak miła...\stoprefrain

\vfill
\bTABLE
    \bTR\bTD t.:\eTD\bTD \quote{Śpiewniczek}, ks.~Jan Siedlecki CM (1878~r.)\eTD\eTR
    \bTR\bTD m.:\eTD\bTD Józef Wygrzewalski\eTD\eTR
\eTABLE


\startsection[title={Wśród nocnej ciszy}] %{{{1
% http://bibliotekapiosenki.pl/Wsrod_nocnej_ciszy_%28koleda%29

\startstanza
Wśród nocnej ciszy głos się rozchodzi:
Wstańcie, pasterze, Bóg się wam rodzi!
Czem prędzej się wybierajcie,
do Betlejem pośpieszajcie, przywitać Pana.\stopstanza

\startstanza
Poszli, znaleźli Dzieciątko w~żłobie
z~wszystkimi znaki danymi sobie.
Jako Bogu cześć Mu dali,
a~witając zawołali z~wielkiej radości:\stopstanza

\startstanza
Ach, witaj Zbawco z~dawna żądany,
cztery tysiące lat wyglądany.
Na Ciebie króle, prorocy
czekali a~Tyś tej nocy nam się objawił.\stopstanza

\startstanza
I~my czekamy na Ciebie, Pana,
a~skoro przyjdziesz na głos kapłana
padniemy na twarz przed Tobą
wierząc żeś jest pod osłoną chleba i~wina.\stopstanza

\vfill
% FIXME
\bTABLE
    \bTR\bTD t.~i~m.:\eTD\bTD \quote{Dodatek II i~III do Śpiewnika kościelnego},\eTD\eTR
    \bTR\bTD\eTD\bTD ks.~M.~Mioduszewski CM (1853~r.)\eTD\eTR
\eTABLE


\startsection[title={W~żłobie leży}] %{{{1
% http://bibliotekapiosenki.pl/W_zlobie_lezy

\startstanza
W~żłobie leży, któż pobieży
kolędować Małemu?
Jezusowi, Chrystusowi,
dziś nam narodzonemu?
Pastuszkowie przybywajcie,
Jemu wdzięcznie przygrywajcie,
jako Panu naszemu.\stopstanza

\startstanza
My zaś sami z~piosneczkami
za wami pospieszymy,
A~tak tego, maleńkiego,
niech wszyscy zobaczmy:
Jak ubogo narodzony,
płacze w~stajni położony,
więc go dziś ucieszymy.\stopstanza

\startstanza
Naprzód tedy, niechaj wszędy
zabrzmi świat w~wesołości,
że posłany nam jest dany
Emmanuel w~niskości!
Jego tedy przywitajmy,
z~aniołami zaśpiewajmy:
Chwała na wysokości!\stopstanza

\startstanza
Witaj, Panie, cóż się stanie,
że rozkosze niebieskie
opuściłeś a~zstąpiłeś
na te niskości ziemskie?
Miłość Moja to sprawiła,
by człowieka wywyższyła
pod nieba emiprejskie.\stopstanza

\startstanza
Czem w~żłobeczku, nie w~łóżeczku,
na siankuś położony?
Czem z~bydlęty, nie z~Panięty,
w~stajni jesteś złożony?
By człek sianu przyrównany,
grzesznik bydlęciem nazwany,
przeze Mnie był zbawiony.\stopstanza

\fillersmall
\bTABLE
    \bTR\bTD t.~i~m.:\eTD\bTD w~kancjonałach benedyktynek staniąteckich (1707~r.)\eTD\eTR
\eTABLE



\startsection[title={Z~narodzenia Pana}] %{{{1
% http://bibliotekapiosenki.pl/Z_narodzenia_Pana
\author{Mioduszewski, ks. Michał, CM+„Śpiewnik kościelny”}

\startstanza
Z~narodzenia Pana dzień dziś wesoły,
wyśpiewują chwałę Bogu żywioły.
Radość ludzi wszędzie słynie,
Anioł budzi przy dolinie
pasterzów, co paśli pod borem woły.\stopstanza

\startstanza
Wypada wśród nocy ogień z~obłoku,
dumają pasterze w~takim widoku.
Każdy pyta: co się dzieje?
Czy nie świta? Czy nie dnieje?
Skąd ta łuna bije tak miła oku?\stopstanza

\startstanza
Ale gdy Anielskie głosy słyszeli,
zaraz do Betlejem prosto bieżeli.
Tam witali w~żłobie Pana,
poklękali na kolana
i~oddali dary, co z~sobą wzięli.\stopstanza

\startstanza
Odchodzą z~Betlejem pełni wesela,
że już Bóg wysłuchał próśb Izraela,
gdy tej nocy to widzieli,
co prorocy widzieć chcieli;
w~ciele ludzkim Boga i~Zbawiciela.\stopstanza

\startstanza
I~my z~pastuszkami dziś się radujmy,
chwałę z~Aniołami wraz wyśpiewujmy.
Bo ten Jezus z~nieba dany,
weźmie nas między niebiany,
tylko Go z~całego serca miłujmy.\stopstanza

\vfill
\bTABLE
    \bTR\bTD t.~i~m.:\eTD\bTD \quote{Dodatek do Śpiewnika kościelnego},\eTD\eTR
    \bTR\bTD\eTD\bTD ks. Michał Mioduszewski CM (1842~r.)\eTD\eTR
\eTABLE


% }}}

% Dodatek {{{1
%\page[yes]\strut
%\page[empty]
\page[odd]\strut
\setupheader[state=high]
\setupfooter[state=high]
\starttitle[title={Dodatek}]

\page[empty,odd]


\startsection[title={A~czemuż mój Jezus}] %{{{1
% http://bibliotekapiosenki.pl/A_czemuz_moj_Jezus_tak_ubogo_lezy
\author{Mioduszewski, ks. Michał, CM+„Śpiewnik kościelny”}
\author{Niemcy}

\startstanza
A~czemuż mój Jezus tak ubogo leży?
Ani po królewsku, ni w~drogiej odzieży?
Znać dlatego, by grzesznika,
czartowskiego niewolnika,
od piekła wybawił, przez ubóstwo zbawił.\stopstanza

\startstanza
Czemuż nie w~pałacach rodzi się Dziecina,
wszak świat, niebo, ziemia, Jego jest dziedzina,
ale w~stajence ubogiej,
na sianeczku w~ten mróz srogi?
W~kamiennym żłóbeczku zimno Paniąteczku.\stopstanza

\startstanza
A~cóż za dworzany ma Boskie Dzieciątko,
Syn Ojca Wiecznego, małe Pacholątko?
Osieł z~wołem to dworzany,
zważ człeku, Pana nad Pany,
bydlęta Mu służą, jak Bogu posłużą.\stopstanza

\vfill
\bTABLE
    \bTR\bTD t.:\eTD\bTD \quote{Śpiewnik kościelny}, ks. Michał Mioduszewski CM (1838~r.)\eTD\eTR
    \bTR\bTD m.:\eTD\bTD niemiecka (XVI~w.)\eTD\eTR
\eTABLE


\startsection[title={Ach witajże pożądana}] %{{{1
% http://bibliotekapiosenki.pl/Ach_witajze_pozadana
\author{Mioduszewski, ks. Michał, CM+„Śpiewnik kościelny”}
\author{Siedlecki, ks. Jan, CM}

\startstanza
Ach, witajże, pożądana perło droga z~nieba,
gdy cały świat upragniony anielskiego chleba!
W~ciele ludzkiem Bóg jest skryty,
na pokarm ludziom obfity;
Ciałem karmi, Krwią napoi,
by człowieka w~chwale swojej
między wybranemi policzył.
 
Niedośćże to Boskie dziecię żeś na świecie z~nami?
Ale jeszcze zimno cierpisz między bydlętami!
Malusieńki Jezu w~żłobie,
co za wielka miłość w~Tobie!
Czyliż nie są wielkie dziwy:
w~ludzkiem ciele Bóg prawdziwy
przyszedł na zbawienie człowieka.\stopstanza

\startstanza
O~miłości niepojęta, jakżeś wielka była!
lżeś się tu z~niebieskiego tronu sprowadziła,
a~to do pustej szopiny,
o~niesłychane nowiny!
Ach pokorny baraneczku,
Twój odpoczynek w~żłobeczku,
z~dalekiej podróży niebieskiej.\stopstanza

\startstanza
Niech Ci, Jezu, będą dzięki za twe narodzenie,
bo przez nie zacząłeś nasze sprawować zbawienie.
Miłość która to sprawiła,
iż Cię do nas sprowadziła,
niech swą iskrą nas zapali
abyśmy cię miłowali
teraz i~bez końca w~wieczności.\stopstanza

\vfill
\bTABLE
    \bTR\bTD t.:\eTD\bTD \quote{Dodatek do Śpiewnika kościelnego}, ks. M. Mioduszewski CM (1841~r.)\eTD\eTR
    \bTR\bTD m.:\eTD\bTD \quote{Śpiewniczek}, ks. Jan Siedlecki CM (1928~r.)\eTD\eTR
\eTABLE


\startsection[title={Nad Betlejem w~ciemną noc}] %{{{1
\author{Francja}

\startstanza
Nad Betlejem w~ciemną noc
śpiewał pieśń Aniołów chór.
Ich radosny, cudny głos
odbijało echo gór.
Gloria in excelsis Deo!\stopstanza

\startstanza
Pastuszkowie jaką pieśń
słyszeliście nocy tej?
Jakaż to radosna wieść
była tam natchnieniem jej?
Gloria in excelsis Deo!\stopstanza

\startstanza
Do Betlejem prędko śpiesz,
zostaw stada pośród gór,
gdyż anielska głosi wieść,
że się tam narodził Król.
Gloria in excelsis Deo!\stopstanza

\startstanza
W~twardym żłobie leży tam
Jezus, nieba, ziemi Pan.
Chciejmy Mu w~pokorze wznieść
uwielbienie, chwałę cześć.
Gloria in excelsis Deo!\stopstanza

\vfill
\bTABLE
    \bTR\bTD t.:\eTD\bTD \quote{Choix de cantiques offerts aux éleves des écoles chrétiennes} (1846~r.)\eTD\eTR
    \bTR\bTD\eTD\bTD oraz \quote{Choix de cantiques sur des airs nouveaux} (1848~r.)\eTD\eTR
    \bTR\bTD m.:\eTD\bTD francuska (XVIII~w.)\eTD\eTR
\eTABLE


\startsection[title={Nie było miejsca dla Ciebie}] %{{{1
% http://bibliotekapiosenki.pl/Nie_bylo_miejsca_dla_ciebie
\author{Jeż, o. Mateusz, SJ}
\composer{Łaś, o. Józef, SJ}

\startstanza
Nie było miejsca dla Ciebie
w~Betlejem w~żadnej gospodzie
i~narodziłeś się, Jezu,
w~stajni, w~ubóstwie i~chłodzie.\stopstanza

\startstanza
Nie było miejsca, choć zszedłeś
jako Zbawiciel na ziemię,
by wyrwać z~czarta niewoli
nieszczęsne Adama plemię.\stopstanza

\startstanza
Nie było miejsca, choć chciałeś
ludzkość przytulić do łona
i~podać z~krzyża grzesznikom
zbawcze, skrwawione ramiona.\stopstanza

\startstanza
Nie było miejsca, choć szedłeś
ogień miłości zapalić
i~przez swą mękę najdroższą
świat od zagłady ocalić.\stopstanza

\startstanza
Gdy liszki mają swe jamy
i~ptaszki swoje gniazdeczka,
dla Ciebie brakło gospody,
Tyś musiał szukać żłóbeczka.\stopstanza

\startstanza
A~dzisiaj czemu wśród ludzi
tyle łez, jęków, katuszy?
Bo nie ma miejsca dla Ciebie
w~niejednej człowieczej duszy.\stopstanza

\vfill
\bTABLE
    \bTR\bTD t.:\eTD\bTD o. Mateusz Jeż SJ (1932~r.)\eTD\eTR
    \bTR\bTD m.:\eTD\bTD o. Józef Łaś SJ; „Największa kantyczka z~nutami \eTD\eTR
    \bTR\bTD\eTD\bTD na 2-3 głosy”, Józef Albin Gwoździkowski (1938~r.)\eTD\eTR
\eTABLE


\startsection[title={Północ już była}] %{{{1
% http://bibliotekapiosenki.pl/Polnoc_juz_byla
\author{karmelitanki}
\author{Szlichtyn, Jan}

\startstanza
Północ już była, gdy się zjawiła
nad bliską doliną jasna łuna,
którą zoczywszy i~zobaczywszy
krzyknął mocno Wojtek na Szymona:
Szymonie kochany, znak to niewidziany,
że całe niebo czerwone!
Na braci zawołaj, niechaj wstawają,
Kuba i~Mikołaj niech wypędzają
barany i~capy, owce, kozły, skopy
zamknione.\stopstanza

\startstanza
Na te wołania z~smacznego spania
porwał się Stach z~Grześkiem i~spadł z~broga.
Maciek truchleje, od strachu mdleje,
woła: uciekajcie, ach dla Boga!
Grześko nogę złomał, Stach na nogę chromał,
bo ją w~kolanie wywinął.
Oj, oj, oj, oj, oj, oj --- Pawełek woła --
uciekaj, dlaboga, gore stodoła;
pogorzały szopy i~pszeniczne snopy,
jagnięta.\stopstanza

\startstanza
Leżąc w~stodole, patrząc na pole,
ujrzał Bartos stary anioły,
którzy wdzięcznymi głosami swymi,
okrzyknęli ziemskie padoły:
Na niebie niech chwała Bogu będzie trwała,
a~ludziom pokój na ziemi.
Pasterze wstawajcie, witajcie Pana!
Pokłon Mu oddajcie, wziąwszy barana.
Skoczno Mu zagrajcie, głosy zaśpiewajcie zgodnemi.\stopstanza

\vfill
\bTABLE
    \bTR\bTD t.:\eTD\bTD w~kantyczkach karmelitańskich (XVIII~w.)\eTD\eTR
    \bTR\bTD\eTD\bTD oraz rozsz.~\quote{Kantyczek pieśni nabożnych} Jana Szlichtyna (1785~r.)\eTD\eTR
    \bTR\bTD m.:\eTD\bTD autor nieznany\eTD\eTR
\eTABLE


\startsection[title={Przystąpmy do szopy}] %{{{1
% http://bibliotekapiosenki.pl/Przystapmy_do_szopy
\author{Flasza, Tomasz}

\startstanza
Przystąpmy do szopy,
uściskajmy stopy
Jezusa narodzonego,
który swoje Bóstwo
wydał na ubóstwo
dla zbawienia naszego.\stopstanza

\startstanza
Zawitaj Zbawco narodzony
z~Przeczystej Panienki;
gdzie berło, gdzie Twoje korony,
Jezu malusieńki?\stopstanza

\startstanza
Ten, co wszechświat dzierży
w~żłobie dzisiaj leży,
ludzkiej pomocy czeka.
Jezus, Bóg wcielony,
w~żłobie położony
dla zbawienia człowieka\stopstanza

\startstanza
O~Boże, bądźże pochwalony
za Twe narodzenie!
Racz zbawić ludzki ród zgubiony,
daj duszy zbawienie!\stopstanza

\vfill
\bTABLE
    \bTR\bTD t.~i~m.:\eTD\bTD \quote{Śpiewnik kościelny katolicki}, Tomasz Flasza (1903-1909~r.)\eTD\eTR
\eTABLE


\startsection[title={Szczęśliwa kolebko}] %{{{1
\author{Mioduszewski, ks. Michał, CM+„Śpiewnik kościelny”}

\startstanza
Szczęśliwa kolebko, szczęśliwy żłobie,
ten Bóg co świat stworzył położon w~tobie:
niebem nieogarniony,
w~tobie wszystek zamkniony.\stopstanza

\startstanza
Kolebko niebieska, żłobku ubogi,
w~tobie jest zawarty ten klejnot drogi,
z~nieba światu spuszczony,
złotem nieoceniony.\stopstanza

\startstanza
Gdy twoją uważam piękną ozdobę,
ceniąc która leży w~żłobie osobę:
nikną wszystkie urody,
miękkie świata wygody.\stopstanza

\startstanza
Za nic złotogłowy, za nic szkarłaty,
kiedy patrzę na te ubogie płaty,
któremi Cię związała
Matka, innych nie miała.\stopstanza

\startstanza
Zasypiaj szczęśliwie w~tej kolebeczce.
Przyśpiewuj matuniu tej Dziecineczce.
Niechaj zasypia wdzięcznie
Jezus mój i~bezpiecznie.\stopstanza

\vfill
\bTABLE
    \bTR\bTD t.~i~m.:\eTD\bTD \quote{Śpiewnik kościelny}, ks. Michał Mioduszewski CM (1838~r.)\eTD\eTR
\eTABLE


\startsection[title={Święta Panienka, Syna usypiała}] %{{{1
\author{Jarewicz, Hanna}
\composer{Maklakiewicz, Jan}

\startstanza
Święta Panienka Syna usypiała,
drżącego z~zimna siankiem otulała.
Tuli, tuli, moje Ubożątko,
tuli, tuli, moje Ty Dzieciątko.\stopstanza

\startstanza
Śpi już spokojnie w~żłóbku kolebeczce,
śpiewa Matusia Świętej Dziecineczce.
Luli, luli, mały mój Synaczku,
luli, luli, mój Ty Jedynaczku.\stopstanza

\fillersmall
\bTABLE
    \bTR\bTD t.:\eTD\bTD Hanna Jarewicz\eTD\eTR
    \bTR\bTD t.~i~m.:\eTD\bTD Jan Maklakiewicz (1899-1954~r.)\eTD\eTR
\eTABLE

%}}}

\doifmode{indices}{
    \page[yes]
    \blank[big]
    \subject{Skorowidz autorów, tradycji i~tłumaczy}
    \placeregister[author]

    \page[yes]
    %\blank[big]
    \subject{Skorowidz kompozytorów}
    \placeregister[composer]
}

% vim: ts=4 sw=4 fdm=marker
